\documentclass{article}
%packages
\usepackage{graphicx}
\usepackage{epigraph}
\usepackage[T1]{fontenc}
\usepackage[utf8]{luainputenc}
\usepackage[compat=1.1.0]{tikz-feynman}
\usepackage{amsthm}
\usepackage{amsmath}
\usepackage[font={small,it}]{caption}
%\usepackage[a5paper, total={4.5in, 7in}]{geometry} %formato libro piccolo
\usepackage[a4paper, total={7.5in, 10.35in}]{geometry} %formato libro A4
\usepackage{hyperref}
\usepackage{color,soul}
\usepackage{amsfonts}
\usepackage{amssymb}
\usepackage{imakeidx}
\usepackage{enumerate}
\usepackage{cancel}
\usepackage{physics}
\makeindex[options=-s mystyle.ist]
\usepackage{simplewick}
%\usepackage{empheq}
\usepackage{tikz}
\usepackage{graphicx}
\usepackage{braket}
\graphicspath{ {./images/} }
\usepackage{titlesec}
\usepackage{fancyhdr}
\usepackage{subcaption}
\usepackage{kpfonts}
\usepackage{listings}
\usepackage{xcolor}
% bibliography
\usepackage[
    backend=biber,
    style=alphabetic,
    sorting=ynt
]{biblatex}
\addbibresource{bibliography/bibliography.bib}
% algorithm
\usepackage{algorithm}
\usepackage[noend]{algpseudocode}%
% multicolumns
\usepackage{multicol}

\title{Motorize a 1980 dad's telescope}

\author{Sebastiano Cocchi \& Stefano Cocchi}

\date{\today}

\begin{document}
    
    \maketitle

    \begin{abstract}
        A 1980 (old and dusty) equatorial telescope is converted to an up-to-date, motorized and computer-connected telescope.
        We, me and my father, illustrate all the transformation steps from an old, dusty and unused telescope into an optimal tool for astrophotography.
    \end{abstract}

    \tableofcontents

    \begin{multicols}{2}
        \section{Telescope Description}
        The starting point of the project if of course the telescope.
        In our garage, for many years, a 1980 Urania telescope has eaten a lot of dust.
        The telescope's mirror resent of years in humidity and temperature jumps in the garage.
        In the beginning, we have cleaned the silvered-mirror with soap and water, but the silver seemed to be a bit compromised.
        We do not talk long about this telescope, since we have soon substituted it with a brand-new Skywatcher Quattro.
        The latter is placed on the Urania mount, since it is still a nice mount and, in our advice, has still not surpassed robustness.
        Indeed, the mount is a very robust equatorial and motorized (still works!) mount.
        
        For our money, but most importantly for our fun and entertainment, we decided to modernize our old telescope.

        \subsection{Urania telescope}
        We briefly add the specifics of the old Urania telescope, as a sort of respect for many years of honorable work before the deep dark garage.

        The telescope is a Urania C.R.T NX 155.
        \\
        \begin{minipage}{0.5\textwidth}
            \centering
            \begin{tabular}{c|c}
                Specific name & value \\
                \hline
                aperture (mm) & \\
                focal length & \\
                magnification & \\
                focal ratio & \\
                resolving power & 
            \end{tabular}
            \captionof{table}{Urania C.R.T NX 155.}
        \end{minipage}
        \\
        \hl{mettere foto}

        \subsection{Skywatcher 8P Quattro telescope}
        Skywatcher 8P Quattro Newtonian telescope offer an optimal astrophotography performance.
        For this reason we have decided to substitute the Urania telescope with this brand-new Skywatcher telescope.
        \\
        \begin{minipage}{0.5\textwidth}
            \centering
            \begin{tabular}{c|c}
                Specific name & value \\
                \hline
                type & Riflettore \\
                technique & Newton  \\
                material & Carbon  \\
                weight (kg) & 8.0 \\
                aperture (mm) & 200 \\
                focal length (mm) & 800 \\
                focal & f/4 \\
                resolution power & 0.58 \\
                limit magnitude value (mag) & 13.3 \\
                collect light & 820 \\
                magnification & 400 \\
                Mirror Treatment & Aluminum Coating \\
                Focuser & Crayford dual-speed 50.8 / 31.8 
            \end{tabular}
            \captionof{table}{Skywatcher 8P Quattro}
        \end{minipage}
        \\
        \hl{mettere foto}

        \subsection{Telescope mount}
        The telescope is place onto an aeronautic Aluminum tripod equatorial mount.
        \hl{quanto pesa? }
        \hl{ metti immagine}
        Starting from the bottom, from a central post, three pods of 30cm depart from the center. Each one has a wheel which permits the structure to move freely and then to fix the position using stops.
        The central post terminates with the second post with an inclination equal to the Earth's ecliptic \(23.43^{\circ} = 23^{\circ} 26'\).

        This axis must be aligned with the Polar star (labelling the North).
        In this way, a 3W electric motor can follow the sky movement.

        Departing from this second axis, a two-arms fork is free to rotate around two degrees-of-freedom defining the right ascension (RA) and the declination (DEG).
        The two arms are separated by the distance \(d = 15\)mm which is the Urania telescope aperture.

        \section{Telescope substitution}
        Passing from the Urania telescope to Skywatcher telescope we have faced the problem of how to insert the latter in the telescope mount.



        \section{Optics list}

    \end{multicols}
\end{document}
\section{Telescope description}
\label{sec:telescope-description}
The starting point of the project is of course the telescope.
In our garage, for many years, a 1980 Urania telescope has eaten a lot of dust.
The telescope's mirror resent of years in humidity and temperature jumps in the garage.
In the beginning, we have cleaned the silvered-mirror with soap and water, but the silver still seemed to be a bit compromised.
We do not talk long about this telescope, since we have soon substituted it with a brand-new Skywatcher Quattro.
The latter is placed on the Urania mount, since it is still a nice mount and, in our advice, has still not surpassed robustness.
Indeed, the mount is a very heavy (telescope and mount totally weight 20kg!) equatorial and motorized (still works!) mount.

For our money, but most importantly for our fun and entertainment, we decided to modernize our old telescope.

\begin{figure*}
    \begin{subfigure}[t!]{0.33\textwidth}
        \centering
        \includegraphics[scale=0.35]{urania_upper.jpg}
        \caption{Urania telescope and mount.}
        \label{fig:urania_telescope_mount}
    \end{subfigure}
    \begin{subfigure}[t!]{0.33\textwidth}
        \centering
        \includegraphics[scale=0.2]{newton-quattro-200-sky-watcher.jpg}
        \caption{Skywatcher Quattro telescope.}
        \label{fig:skywatcher_telescope_mount}
    \end{subfigure}
    \begin{subfigure}[t!]{0.33\textwidth}
        \centering
        \includegraphics[scale=0.5]{DEC_sede.jpg}
        \caption{The mechanism built to insert Skywatcher's telescope into the Urania robust mount.}
        \label{fig:piastra_DEC}
    \end{subfigure}
    \caption{The transition between the old telescope to the new telescope: (a) the starting point telescope, (b) the new telescope, (c) and (d) the telescope seat used for the substitution.}
    \label{fig:telescope-subst}
\end{figure*}

\subsection{Urania telescope}
We briefly add the specifics of the old Urania telescope, as a sort of respect for many years of honorable work before the deep dark in the garage.

The telescope is a Urania C.R.T. NX 155, as the one in figure \ref{fig:urania_telescope_mount}.
\\
\\
\begin{minipage}{0.5\textwidth}
    \centering
    \begin{tabular}{c|c}
        \hline
        \textbf{Specific's name} & \textbf{value} \\
        \hline
        type & reflector \\
        technique & Newton  \\
        material & PVC  \\
        weight (kg) & 10 \\
        aperture (mm) & 155 \\
        focal length (mm) & 1000 \\
        focal & f/6.5 \\
        resolution power & 0.8 \\
        limit magnitude value (mag) & 13.6 \\
        Mirror Treatment & Silica monoxide \\
        \hline
    \end{tabular}
    \captionof{table}{Urania C.R.T. NX 155 specifics.}
\end{minipage}

\begin{figure*}
    \begin{subfigure}[t!]{.23\textwidth}
        \centering
        \includegraphics[scale=.4]{cuscinetto-slim.jpg}
        \caption{Internal bearing mounted on one fork arm.}
        \label{fig:cuscinetto-slim}
    \end{subfigure}
    \begin{subfigure}[t!]{.23\textwidth}
        \centering
        \includegraphics[scale=.4]{cuscinetto-dec.jpg}
        \caption{External bearing mounted on one fork arm.}
        \label{fig:cuscinetto-DEC}
    \end{subfigure}
    \begin{subfigure}[t!]
        {0.5\textwidth}
        \centering
        \includegraphics[scale=0.75]{DEC_piastra_dimensioni.png}
        \caption{Schematic view of the telescope mount insertion.
        This represents only the installation build with some squared metal bars upon which the two white loops (which hold the telescope tube) are fixed and are not illustrated in this scheme. The two holes in each arm serve to fix the structure on the mount. }
        \label{fig:DEC_piastra_dimensioni}
    \end{subfigure}
    \caption{Telescope seat details.}
    \label{fig:telescope-substitution}
\end{figure*}

\subsection{Telescope's mount}
The telescope is place onto an aeronautic Aluminum tripod equatorial mount.
\\
\\
\begin{minipage}{.4\textwidth}
    \begin{tabular}{c|c}
        \hline
        \textbf{Specific's name} & \textbf{value} \\
        \hline
        weight (kg) & 25kg \\
        type & fork \\
        material & Aluminum alloy \\
        RA axis diameter (mm) & 30 \\
        RA axis material & cadmium steel \\
        RA motor & 3W synchronous \\
        \hline
    \end{tabular}
    \captionof{table}{Urania's mount specifics.}
    \label{tab:mount}
\end{minipage}

Starting from the bottom: three pods of 45cm depart from a central post. Each one has a wheel which permits the structure to move freely and then to fix the position using stops.
The central post terminates with the second axis holder whose inclination can be adjusted using a big screw.
Using a digital inclinometer we fix the axis to be at \(45.75'\) with respect to the ground.

This axis must be aligned with the Polar star (labelling the North).
In this way, a 3W synchronous motor can follow the sky movement.

Departing from this second axis, a two-arms fork is free to rotate around two degrees-of-freedom defining the right ascension (RA) and the declination (DEC).
The two arms are separated by the distance \(d = 260\)mm which is enough to fit the Urania telescope aperture.
\subsection{Skywatcher 8P Quattro telescope}
Skywatcher 8P Quattro Newtonian telescope (figure \ref{fig:skywatcher_telescope_mount}) offers an optimal astrophotography performance.
For this reason we have decided to substitute the Urania telescope with this brand-new Skywatcher telescope.
\\
\\
\begin{minipage}{0.5\textwidth}
    \centering
    \begin{tabular}{c|c}
        \hline
        \textbf{Specific's name} & \textbf{value} \\
        \hline
        type & reflector \\
        technique & Newton  \\
        material & Carbon  \\
        weight (kg) & 8.0 \\
        aperture (mm) & 200 \\
        focal length (mm) & 800 \\
        focal & f/4 \\
        resolution power & 0.58 \\
        limit magnitude (mag) & 13.3 \\
        collect light & 820 \\
        magnification & 400 \\
        Mirror Treatment & Aluminum Coating \\
        Focuser & Crayford dual-speed 50.8/31.8 \\
        \hline
    \end{tabular}
    \captionof{table}{Skywatcher 8P Quattro}
    \label{tab_skywatcher_quattro}
\end{minipage}
\\
Since the Skywatcher telescope does not fit in the fork, we have thought to build a "saddle" onto which placing the telescope.
The specifics of this installation are shown in the following section.

\subsection{Telescope substitution: from Urania's tube to Skywatcher's tube}
Passing from the Urania telescope to Skywatcher telescope we have faced the problem of how to insert the latter in the telescope mount.
Indeed, since Skywatcher's telescope diameter is 200mm it does not fit inside the mount fork.

Our solution is to insert a seat in which to place the telescope, see figure \ref{fig:piastra_DEC}.
The barycenter of the telescope is not centered with the DEC axis, thus we have settled a post capable of holding weights to balance the forces, see figure \ref{fig:DEC_piastra_dimensioni}.

The scheme with distances is visible in figure \ref{fig:DEC_piastra_dimensioni}.

Lastly, to enhance the fluidity of motion and reduce the backlash, the rotation mechanism is enriched with a system of two bearings for each fork arm, one is place inside, see figure \ref{fig:cuscinetto-slim}, and one is fixed externally on the mount, see figure \ref{fig:cuscinetto-DEC}.

The total weight of the mount with this modification is attested on 30kg.

% \begin{figure*}
%     \begin{subfigure}[h!]
%         {.5\textwidth}
%         \centering
%         \includegraphics[scale=.45]{cuscinetto-slim.jpg}
%         \caption{Internal bearing mounted on one fork arm.}
%         \label{fig:cuscinetto-slim}
%     \end{subfigure}
%     \begin{subfigure}[h!]
%         {.5\textwidth}
%         \centering
%         \includegraphics[scale=.45]{cuscinetto-dec.jpg}
%         \caption{External bearing mounted on one fork arm.}
%         \label{fig:cuscinetto-DEC}
%     \end{subfigure}
%     \caption{Bearings mounted on the fork.}
%     \label{fig:bearings}
% \end{figure*}